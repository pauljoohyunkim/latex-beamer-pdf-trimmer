% arara: pdflatex

\documentclass{beamer}
\usetheme{Madrid}

\usepackage[]{amsmath}
\usepackage{amssymb}
\usepackage{amsthm}

\newcommand{\diag}[1]{\text{diag}\,#1}

\title{Amazing Presentation About Fibonacci Sequence}
\author{Paul Kim}

\begin{document}

\frame{\titlepage}

\begin{frame}
    \frametitle{Fibonacci Stuff}
    \begin{definition}
        Fibonacci sequence $\left\{ F_n \right\}$ is defined as:
        \begin{align*}
            F_1 &= 1 \\
            F_2 &= 1 \\
            \forall n \in \mathbb{Z}^{\geq 0}: F_{n+2} &= F_{n} + F_{n+1}
        \end{align*}
    \end{definition}
    \onslide<2->
    {
        \textbf{Goal}: Finding the general formula for the Fibonacci sequence!
    }
\end{frame}

\begin{frame}
    \frametitle{Derivation}
    Here is a slick way to get a general formula for Fibonacci sequence.
    \onslide<2->
    {
        Rewrite the recurrence equation as:
        \begin{align*}
            \begin{pmatrix}
                F_{n+2} \\
                F_{n+1}
            \end{pmatrix}
            &= 
            \begin{pmatrix}
                1 & 1 \\
                1 & 0
            \end{pmatrix}
            \begin{pmatrix}
                F_{n+1} \\
                F_n
            \end{pmatrix}
        \end{align*}
        (I know, the second equation is redundant, but bare with me\dots)
    }
    \onslide<3->
    Now, by eigendecomposition of the matrix:
    \begin{equation*}
        \begin{pmatrix}
            1 & 1 \\
            1 & 0
        \end{pmatrix}
        =
        V D V^T
    \end{equation*}
    where $D = \diag{\left( \phi, \psi \right)}$ is the diagonal matrix with entries being the two roots of
    $\lambda^2 - \lambda - 1 = 0$.
\end{frame}

\begin{frame}
    So the recurrence equation can be written as:
        \begin{align*}
            \begin{pmatrix}
                F_{n+2} \\
                F_{n+1}
            \end{pmatrix}
            &= 
            V D V^T
            \begin{pmatrix}
                F_{n+1} \\
                F_n
            \end{pmatrix}
        \end{align*}
        \onslide<2->
        {
            By induction, one deduces:
            \begin{align*}
                \begin{pmatrix}
                    F_{n+1} \\
                    F_n
                \end{pmatrix}
                &=
                \left( VDV^T \right)^{n}
                \begin{pmatrix}
                    F_{1} \\
                    F_{0}
                \end{pmatrix}
                \\
                &= 
                V D^n V^T
                \begin{pmatrix}
                    F_{1} \\
                    F_{0}
                \end{pmatrix}
            \end{align*}
            where the last equality comes from $V$ being orthogonal and $D$ being diagonal.
        }
        \onslide<3->
        {
            It is easy to now acquire the general form of $F_n$.
            (It is left as an exercise.)
        }
\end{frame}

\end{document}
